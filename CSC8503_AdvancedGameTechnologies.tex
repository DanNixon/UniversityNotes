\documentclass[a4paper]{article}
\def\DOCTITLE{CSC8503 Advanced Game Technologies}
% Set document attributes
\title{\DOCTITLE}

\usepackage{fullpage}
\usepackage{scrextend}
\usepackage{titlesec}
\usepackage{fancyhdr}
\usepackage{amsmath}
\usepackage{amssymb}
\usepackage[section]{placeins}
\usepackage{booktabs}
\usepackage{hyperref}
\usepackage{tikz}
\usepackage{graphicx}
\usepackage{minted}
\usepackage{subcaption}

% Setup headers and footers
\pagestyle{fancy}
\lhead{}
\chead{\DOCTITLE}
\rhead{}
\rfoot{}
\cfoot{\thepage}
\lfoot{}

% New page for each section
\newcommand{\sectionbreak}{\clearpage}

% Set header and footer sizes
\renewcommand{\headrulewidth}{0.4pt}
\renewcommand{\footrulewidth}{0.4pt}
\setlength{\headheight}{15.2pt}
\setlength{\headsep}{15.2pt}

\setlength{\parskip}{5pt plus 1pt minus 1pt}
\setlength{\parindent}{0pt}

% Newline after paragraph
\newcommand{\Para}[1]{\paragraph{#1}\mbox{}}

% Stuff used in cryptography notes
\newcommand{\Forall}{\;\forall\;}
\newcommand{\Mod}{\: mod \:}

% Stuff used in distributed systems notes
\newcommand{\happenbefore}{\rightarrow}
\newcommand{\orderbefore}{\Rightarrow}
\newcommand{\clockcond}{\leadsto}
\newcommand{\RArrow}{$\rightarrow$}

\def\checkmark{\tikz\fill[scale=0.4](0,.35) -- (.25,0) -- (1,.7) -- (.25,.15) -- cycle;}


\begin{document}

\tableofcontents

\vfill
Course material:
\url{https://research.ncl.ac.uk/game/mastersdegree/gametechnologies/}

\section{Physics}

\subsection{Newtonian Dynamics}

\subsubsection{Newton's First law}

\textit{In an intertial reference frame, an object either remains at rest or
continues to move at a constant velocity, unless acted upon by a force.}

\begin{itemize}
  \item
    In a games physics engine forces that would typically dampen the velocity of
    an object in motion (e.g. friction, air resistance, etc.) are abstracted to
    simple damping factors
\end{itemize}

\subsubsection{Newton's Second law}

\textit{In an intertial reference frame, the vector sum of forces on an object
  is equal to the mass of the object multiplied by the acceleration of the
object.}

\[
  F = ma
\]

\subsubsection{Newton's Third law}

\textit{When one body exerts a force on a second body, the second body
  simultaneously exerts a force equal in magnitude and opposite in
direction on the first body.}

\subsubsection{Conservation of Momentum}

\textit{In a closed system, the total momentum is constant.}

\[
  p = mv
\]

In a collision:
\begin{align*}
  p^{i} &= p^{f} \\
  p^{i}_{1} + p^{i}_{2} &= p^{f}_{1} + p^{f}_{2} \\
  m^{i}_{1} v^{i}_{1} + m^{i}_{2} v^{i}_{2} &= m^{f}_{1} v^{f}_{1} + m^{f}_{2} v^{f}_{2}
\end{align*}

\subsubsection{Torque}

\begin{itemize}
  \item
    Result of a force applied to an object a given distance from a pivot point

  \item
    Torque produced by force $F$ at distance from pivot $d$: $\tau = dF$

\end{itemize}

\subsubsection{Inertia}

\begin{itemize}
  \item
    TODO

\end{itemize}

\subsection{Physics Engine}

\begin{itemize}
  \item
    Role of physics engine:

    \begin{enumerate}
      \item[1]
        Move objects

      \item[2]
        Detect collisions

      \item[3]
        Resolve collisions

    \end{enumerate}

\end{itemize}

\subsubsection{Orientation}

\begin{itemize}
  \item
    Stored as quaternion

  \item
    $\Theta = \left(
      xsin\left(\frac{\theta}{2}\right),
      ysin\left(\frac{\theta}{2}\right),
      zsin\left(\frac{\theta}{2}\right),
      cos\left(\frac{\theta}{2}\right)
    \right)$

    where $(x, y, z)$ is the axis of rotation and $\theta$ is the angle of
    rotation

\end{itemize}

\subsubsection{Object state}

\begin{description}
  \item[Position $s$] \hfill
    \begin{itemize}
      \item
        $s = \int v \: dt$

    \end{itemize}

  \item[Linear Velocity $v$] \hfill
    \begin{itemize}
      \item
        $v = \int a \: dt$

    \end{itemize}

  \item[Linear Acceleration $a$] \hfill
    \begin{itemize}
      \item
        $a = \frac{F}{m}$

    \end{itemize}

  \item[Force $F$] \hfill

  \item[Mass $m$] \hfill

  \item[Orientation $\theta$] \hfill
    \begin{itemize}
      \item
        $\theta = \int \omega \: dt$

      \item
        Represented as quaternion

    \end{itemize}

  \item[Angular Velocity $\omega$] \hfill
    \begin{itemize}
      \item
        $\omega = \int \alpha \: dt$

    \end{itemize}

  \item[Angular Acceleration $\alpha$] \hfill
    \begin{itemize}
      \item
        $\alpha = \frac{\tau}{I}$

    \end{itemize}

  \item[Torque $\tau$] \hfill

  \item[Inertia $I$] \hfill

\end{description}

\subsubsection{Physics Representation}

\begin{description}
  \item[Particle] \hfill
    \begin{itemize}
      \item
        TODO

    \end{itemize}

  \item[Rigid Body] \hfill
    \begin{itemize}
      \item
        TODO

    \end{itemize}

  \item[Soft Body] \hfill
    \begin{itemize}
      \item
        TODO

    \end{itemize}

\end{description}

\begin{table}[]
  \centering
  \begin{tabular}{@{}llllll@{}}
    \toprule
               & Velocity   & Angular    & Volume     & Deformation & Collision    \\
    \midrule
    Particle   & \checkmark & \crossmark & \crossmark & \crossmark  & (\crossmark) \\
    Rigid Body & \checkmark & \checkmark & \checkmark & \crossmark  & (\checkmark) \\
    Soft Body  & \checkmark & \checkmark & \checkmark & \checkmark  & (\checkmark) \\
    \bottomrule
  \end{tabular}
  \label{tab:physical_representation_comparison}
  \caption{Comparison of features of different physical representations}
\end{table}
\FloatBarrier

\subsection{Numerical integration}

TODO

\subsection{Constraints}

TODO

\subsection{Collision Detection}

TODO

\subsection{Collision Manifolds}

TODO

\subsection{Collision Response}

TODO

\subsection{Solvers}

TODO

\section{Artificial Intelligence}

TODO

\subsection{Finite State Machine}

TODO

\subsection{Path Finding}

TODO

\section{Networking}

TODO

\section{Massively parallel and Heterogeneous computing}

TODO

\end{document}
