\documentclass[a4paper]{article}
\def\DOCTITLE{CSC3222 Gaming Simulations}
% Set document attributes
\title{\DOCTITLE}

\usepackage{fullpage}
\usepackage{scrextend}
\usepackage{titlesec}
\usepackage{fancyhdr}
\usepackage{amsmath}
\usepackage{amssymb}

% Handle graphics correctly
\ifx\pdftexversion\undefined
\usepackage{graphicx}
% \usepackage[dvips]{graphicx}
\else
\usepackage[pdftex]{graphicx}
\DeclareGraphicsRule{*}{mps}{*}{}
\fi

% Setup headers and footers
\pagestyle{fancy}
\lhead{}
\chead{\DOCTITLE}
\rhead{}
\rfoot{}
\cfoot{\thepage}
\lfoot{}

% New page for each section
% \newcommand{\sectionbreak}{\clearpage}

% Set header and footer sizes
\renewcommand{\headrulewidth}{0.4pt}
\renewcommand{\footrulewidth}{0.4pt}
\setlength{\headheight}{15.2pt}
\setlength{\headsep}{15.2pt}

\setlength{\parskip}{5pt plus 1pt minus 1pt}
\setlength{\parindent}{0pt}

\newcommand{\Forall}{\;\forall\;}
\newcommand{\Mod}{\: mod \:}


\begin{document}

\tableofcontents

\section{Equations and their Solutions}

\subsection{Newton's First Law}

\textit{When viewed in an inertial reference frame, an object is either at rest
or moves at a constant velocity, unless acted upon by an external force.}

\begin{itemize}
  \item
    Objects in a scene that are to be kept in a constant position (e.g. ground)
    should be immune to physics calculations.

  \item
    Movable objects should have a "rest" state in which no physics calculations
    are performed for that object until something will cause it to move.

    This both makes the simulation less computationally intensive and reduced
    the chance of jitter and error caused by two nearly equal floating point
    numbers.
\end{itemize}

\subsection{Newton's Second Law}

\textit{The vector sum of the forces acting on an object is equal to the total
mass of that object multiplied by the acceleration of the object.}

\[
  F = ma
\]

\subsection{Newton's Third Law}

\textit{When one body exerts a force on a second body, the second body
simultaniously exerts a force equal in magnitude and opposite in direction to
that of the first body.}

\subsection{Law of Conservation of Momentum}

\textit{In a closed system, total momentum is consistent.}

\[
  p = mv
\]

Ignoring all other forces if two objects of mass $m_{n}$ travelling towards each
other with momentum $u_{n}$, the total momentum of the system is:
\[
  p_{initial} = m_{1}u_{1} + m_{2}u_{2}
\]

After they collide the two objects will have two new velocities while keeping
the same masses:
\begin{align*}
  p_{final} &= m_{1}v_{1} + m_{2}v_{2} \\
  p_{final} &= p_{initial}
\end{align*}

\subsection{SUVAT equations}

\begin{description}
  \item[$s$] Displacement
  \item[$u$] Initial velocity
  \item[$v$] Final velocity
  \item[$a$] Acceleration
  \item[$t$] Time
\end{description}

\begin{align*}
      v &= u + at \\
      s &= ut + \frac{1}{2}at^{2} \\
      s &= \frac{1}{2} \left(u + v\right) t \\
  v^{2} &= u^{2} + 2as \\
      s &= vt - \frac{1}{2}at^{2}
\end{align*}

Constant acceleration is assumed, this is not an issue in simulation as updates
are calculated in time steps sufficiently small that constant acceleration can
be assumed.

\section{Newtonian Physics}

TODO

\section{Springs}

TODO

\section{Fluid Dynamics}

TODO

\section{Collision Detection and Response}

TODO

\section{Particles}

TODO

\section{Path Finding}

TODO

\section{Choice and States}

TODO

\end{document}
