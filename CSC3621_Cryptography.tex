\documentclass[a4paper]{article}
\def\DOCTITLE{CSC3621 Cryptography}
% Set document attributes
\title{\DOCTITLE}

\usepackage{fullpage}
\usepackage{scrextend}
\usepackage{titlesec}
\usepackage{fancyhdr}
\usepackage{amsmath}
\usepackage{amssymb}

% Handle graphics correctly
\ifx\pdftexversion\undefined
\usepackage{graphicx}
% \usepackage[dvips]{graphicx}
\else
\usepackage[pdftex]{graphicx}
\DeclareGraphicsRule{*}{mps}{*}{}
\fi

% Setup headers and footers
\pagestyle{fancy}
\lhead{}
\chead{\DOCTITLE}
\rhead{}
\rfoot{}
\cfoot{\thepage}
\lfoot{}

% New page for each section
% \newcommand{\sectionbreak}{\clearpage}

% Set header and footer sizes
\renewcommand{\headrulewidth}{0.4pt}
\renewcommand{\footrulewidth}{0.4pt}
\setlength{\headheight}{15.2pt}
\setlength{\headsep}{15.2pt}

\setlength{\parskip}{5pt plus 1pt minus 1pt}
\setlength{\parindent}{0pt}

\newcommand{\Forall}{\;\forall\;}
\newcommand{\Mod}{\: mod \:}


\begin{document}

TODO

\section{Number Theory}

\begin{itemize}
  \item $\mathbb{N}$ denotes positive integer such that $N \in \mathbb{N}^{+}$
  \item $p$ denotes a prime number
  \item $\mathbb{Z}_{N}$ is the set of integers $\{0, 1, 2, \ldots, N-1\}$
\end{itemize}

\subsection{Modular Arithmetic}

Arithmetic module $Z_{m}$ is the set $\{0, \ldots, m-1\}$ with operations
$+$ and $\times$ such that:

\begin{enumerate}
  \item[1] Addition is closed:
           \[\Forall a, b \in Z_{m}, a + b \in Z_{m}\]
  \item[2] Addition is commutative: \\
           \[\Forall a, b \in Z_{m}, a + b = b + a\]
  \item[3] Addition is associative: \\
           \[\Forall a, b, c \in Z_{m}, (a + b) + c = a + (b + c)\]
  \item[4] 0 is an additive identity: \\
           \[\Forall a \in Z_{m}, a + 0 = 0 + a = a\]
  \item[5] Additive inverse of any $a \in Z_{m}$ is $m - a$: \\
           \[\Forall a \in Z_{m}, a + (m - a) = (m - a) + a = 0\]
  \item[6] Multiplication is closed: \\
           \[\Forall a, b \in Z_{m}, ab \in Z_{m}\]
  \item[7] Multiplication is commutative: \\
           \[\Forall a, b \in Z_{m}, ab = ba\]
  \item[8] Multiplication is associative:\\
           \[\Forall a, b, c \in Z_{m}, (ab)c = a(bc)\]
  \item[9] 1 is a multiplicative identity:
           \[\Forall a \in Z_{m}, a \times 1 = 1 \times a = a\]
  \item[10] The distributive property:
           \[\Forall a, b, c \in Z_{m}, (a + b)c = (ac) + (bc) \ \mathrm{and} \ a(b + c) = (ab) + (ac)\]
\end{enumerate}

Items 1-5 establish $Z_{m}$ is an abelian group, items 1-10 establish $Z_{m}$ is
a ring.

\subsection{Greatest Common Denominator}

\[\Forall x, y \in \mathbb{Z}, \exists a, b \in \mathbb{Z} :
a \cdot x + b \cdot y = gcd(x, y)\]

Integers $x$ and $y$ are coprime (relatively prime) if:
\[gcd(x, y) = 1 = a \cdot x + b \cdot y\]

\subsubsection{Euclidean algorithm to compute GCD}

\begin{itemize}
  \item Inputs: integers $x$ and $y$
  \item Output: $gcd(x, y)$
  \item $gcd(x, 1) = x$ as base case 1
  \item $\Forall x, y \in \mathbb{Z}, gcd(x, y) = gcd(y, x \Mod y)$
\end{itemize}

\subsection{Multiplicative Modular Inverse}

Modular inverse of $x$ is denoted by $x^{-1}$.

$y$ is the multiplicative inverse of $x \Mod N$ if $x \cdot y = 1 (\Mod N)$.

$x$ has a multiplicative inverse modulo $N$ iff. $gcd(x, N) = 1$.

Compute using Euclidean algorithm:
\begin{align*}
  gcd(x, N) = 1 &= a \cdot x + b \cdot N \\
  a \cdot x &= 1 \in \mathbb{Z}_{N} \\
  a &= x^{-1} \in \mathbb{Z}_{N} \\
\end{align*}

\subsection{Invertible elements in $\mathbb{Z}_{N}$}

The set of invertible elements is defined by:
\[(\mathbb{Z}_{N})^{*} = \{x \in \mathbb{Z}_{N} : gcd(x, N) = 1\}\]

If $p$ is prime then $(\mathbb{Z}_{p})^{*} = \mathbb{Z}_{p} / {0}$

Set of invertible elements $\mathbb{Z}_{p}$ is cyclic.

$(\mathbb{Z}_{p})^{*}$ is a cyclic group.

\[
  \exists g \in (\mathbb{Z}_{n})^{*} :
  \{1, g, g^{2}, g^{g}, \ldots, g^{p-2}\} = (\mathbb{Z}_{p})^{*}
\]

$g$ is  generator of $(\mathbb{Z}_{p})^{*}$, for example:
\[p = 5: \{1, 3, 3^{2}, 3^{3}\} = \{1, 3, 4, 2\} = (\mathbb{Z}_{5})^{*}\]

\subsection{Solving linear equations}

To solve:
\[a \cdot x + b = 0 \in \mathbb{Z}_{N}\]

\begin{enumerate}
  \item[1] Compute inverse $a^{-1}$
  \item[2] Subtract $b$
  \item[3] Multiply by inverse $a^{-1}$
\end{enumerate}

Solution:
\[x = -b \cdot a^{-1} \in \mathbb{Z}_{N}\]

\end{document}
