\documentclass[a4paper]{article}
\def\DOCTITLE{CSC3121 Distributed Systems}
% Set document attributes
\title{\DOCTITLE}

\usepackage{fullpage}
\usepackage{scrextend}
\usepackage{titlesec}
\usepackage{fancyhdr}
\usepackage{amsmath}
\usepackage{amssymb}
\usepackage[section]{placeins}
\usepackage{booktabs}
\usepackage{hyperref}
\usepackage{tikz}
\usepackage{graphicx}
\usepackage{minted}
\usepackage{subcaption}

% Setup headers and footers
\pagestyle{fancy}
\lhead{}
\chead{\DOCTITLE}
\rhead{}
\rfoot{}
\cfoot{\thepage}
\lfoot{}

% New page for each section
\newcommand{\sectionbreak}{\clearpage}

% Set header and footer sizes
\renewcommand{\headrulewidth}{0.4pt}
\renewcommand{\footrulewidth}{0.4pt}
\setlength{\headheight}{15.2pt}
\setlength{\headsep}{15.2pt}

\setlength{\parskip}{5pt plus 1pt minus 1pt}
\setlength{\parindent}{0pt}

% Newline after paragraph
\newcommand{\Para}[1]{\paragraph{#1}\mbox{}}

% Stuff used in cryptography notes
\newcommand{\Forall}{\;\forall\;}
\newcommand{\Mod}{\: mod \:}

% Stuff used in distributed systems notes
\newcommand{\happenbefore}{\rightarrow}
\newcommand{\orderbefore}{\Rightarrow}
\newcommand{\clockcond}{\leadsto}
\newcommand{\RArrow}{$\rightarrow$}

\def\checkmark{\tikz\fill[scale=0.4](0,.35) -- (.25,0) -- (1,.7) -- (.25,.15) -- cycle;}


\begin{document}

\tableofcontents

\section{RPC semantics}

\begin{table}[h]
  \centering
  \begin{tabular}{@{}llll@{}}
    \toprule
    Message & Type    & Direction   & Remarks                                       \\
    \midrule
    REQ     & Request         & C \RArrow S & Client wants service from server      \\
    REP     & Reply           & S \RArrow C & Server replies to client request      \\
    ACK     & Acknowledgement & (both)      & Previous message arrived successfully \\
    AYA     & Are You Alive?  & C \RArrow S & Check server is functioning           \\
    IAA     & I Am Alive      & S \RArrow C & Server confirms it is functioning     \\
    TA      & Try Again       & S \RArrow C & Server is busy                        \\
    AU      & Address Unknown & S \RArrow C & No server at this address             \\
    \bottomrule
  \end{tabular}
  \caption{Protocol Message Types}
  \label{tab:message_types}
\end{table}

\begin{itemize}
  \item REQ and REP are essential
  \item ACK helps ensure reliability by handling message loss
  \item AYA and IAA allow the client to determine if the server is available if
        it receives a timeout following a REQ or ACK
\end{itemize}

Remote Procedure Calls (RPC) encapsulates the message sending and receiving
between the client and server such that the client code does not have to deal
with it.

Assuming there is some client code $C$ compiled to $C_{obj}$ and server code $S$
compiled to $S_{obj}$, in order for the code to be executed remotely both are
linked to a language specific stub which contains the logic for the remote
communication between client and server.

The stubs for the client and server are produced using an interface definition
$S_{if}$ which describes the interface to the server code $S$.

Basic call procedure:

\begin{enumerate}
  \item[1] Client calls client stub as it would the server code if it was
           compiled locally
  \item[2] Client stub builds the message (including parameter packing) to be
           sent to the server
  \item[3] Message is sent to the server
  \item[4] Server receives the message and passes it to the server stub
  \item[5] Server stub parses the message, unpacks the parameters and calls the
           actual routine
  \item[6] The code executes and the results are packed into a reply message
  \item[7] The reply message is sent to the client
  \item[8] The client stub unpacks the results and returns them to the client
           routine
\end{enumerate}

\subsection{Parameter Passing}

\subsubsection{Methods}

Methods of parameter passing:

\begin{description}
  \item[Pass by value] \hfill \\
    The parameter value is copied to the stack of the executing process.

    Changes to the value of the variable in the called routine do not affect the
    original copy.

  \item[Pass by refernce] \hfill \\
    A reference to the original variable is copied to the stack of the executing
    process.

    When the called routine changes the value of the parameter the original copy
    is also changed.

  \item[Pass by copy-restore (aka value-result)] \hfill \\
    The parameter value is copied to the stack of the executing process, then
    copied back when the routine terminates.

    When the called routine changes the value of the parameter the original copy
    is also changed.

\end{description}

In most situations pass by reference and pass by copy-restore yield identical
results.

In RPC pass by copy-restore is used in place of pass by reference.

\subsubsection{Packing}

TODO

\subsection{Binding}

TODO

\subsection{Failure Handling}

TODO

\section{Clocks and Order}

TODO

\section{Transactions}

For controlling operations on persistent storage resources.

"ACID" properties:

\begin{description}
  \item[Atomicity] \hfill \\
    "All or nothing". \\
    In the event of failure the state must be restored to its condition prior to
    execution.
  \item[Consistency] \hfill \\
    Ensures executions are scheduled to run such that one will not interfere
    with another.
  \item[Independence] \hfill \\
    The effects of an execution (A) must not become visible to another execution
    (B) until after A has successfully completed.
  \item[Durability] \hfill \\
    Results produced by a successful execution are not damaged by a subsequent
    failed execution.
\end{description}

\end{document}
