\documentclass[a4paper]{article}
\def\DOCTITLE{CSC3121 Distributed Systems}
% Set document attributes
\title{\DOCTITLE}

\usepackage{fullpage}
\usepackage{scrextend}
\usepackage{titlesec}
\usepackage{fancyhdr}
\usepackage{amsmath}
\usepackage{amssymb}

% Handle graphics correctly
\ifx\pdftexversion\undefined
\usepackage{graphicx}
% \usepackage[dvips]{graphicx}
\else
\usepackage[pdftex]{graphicx}
\DeclareGraphicsRule{*}{mps}{*}{}
\fi

% Setup headers and footers
\pagestyle{fancy}
\lhead{}
\chead{\DOCTITLE}
\rhead{}
\rfoot{}
\cfoot{\thepage}
\lfoot{}

% New page for each section
% \newcommand{\sectionbreak}{\clearpage}

% Set header and footer sizes
\renewcommand{\headrulewidth}{0.4pt}
\renewcommand{\footrulewidth}{0.4pt}
\setlength{\headheight}{15.2pt}
\setlength{\headsep}{15.2pt}

\setlength{\parskip}{5pt plus 1pt minus 1pt}
\setlength{\parindent}{0pt}

\newcommand{\Forall}{\;\forall\;}
\newcommand{\Mod}{\: mod \:}


\begin{document}

\tableofcontents

\section{RPC semantics}
\label{sec:rpc}

\begin{table}[h]
  \centering
  \begin{tabular}{@{}llll@{}}
    \toprule
    Message & Type    & Direction   & Remarks                                       \\
    \midrule
    REQ     & Request         & C \RArrow S & Client wants service from server      \\
    REP     & Reply           & S \RArrow C & Server replies to client request      \\
    ACK     & Acknowledgement & (both)      & Previous message arrived successfully \\
    AYA     & Are You Alive?  & C \RArrow S & Check server is functioning           \\
    IAA     & I Am Alive      & S \RArrow C & Server confirms it is functioning     \\
    TA      & Try Again       & S \RArrow C & Server is busy                        \\
    AU      & Address Unknown & S \RArrow C & No server at this address             \\
    \bottomrule
  \end{tabular}
  \caption{Protocol Message Types}
  \label{tab:message_types}
\end{table}

\begin{itemize}
  \item REQ and REP are essential
  \item ACK helps ensure reliability by handling message loss
  \item AYA and IAA allow the client to determine if the server is available if
        it receives a timeout following a REQ or ACK
\end{itemize}

Remote Procedure Calls (RPC) encapsulates the message sending and receiving
between the client and server such that the client code does not have to deal
with it.

Assuming there is some client code $C$ compiled to $C_{obj}$ and server code $S$
compiled to $S_{obj}$, in order for the code to be executed remotely both are
linked to a language specific stub which contains the logic for the remote
communication between client and server.

The stubs for the client and server are produced using an interface definition
$S_{if}$ which describes the interface to the server code $S$.

Basic call procedure:

\begin{enumerate}
  \item[1] Client calls client stub as it would the server code if it was
           compiled locally
  \item[2] Client stub builds the message (including parameter packing) to be
           sent to the server
  \item[3] Message is sent to the server
  \item[4] Server receives the message and passes it to the server stub
  \item[5] Server stub parses the message, unpacks the parameters and calls the
           actual routine
  \item[6] The code executes and the results are packed into a reply message
  \item[7] The reply message is sent to the client
  \item[8] The client stub unpacks the results and returns them to the client
           routine
\end{enumerate}

\subsection{Parameter Passing}

\subsubsection{Methods}

Methods of parameter passing:

\begin{description}
  \item[Pass by value] \hfill \\
    The parameter value is copied to the stack of the executing process.

    Changes to the value of the variable in the called routine do not affect the
    original copy.

  \item[Pass by refernce] \hfill \\
    A reference to the original variable is copied to the stack of the executing
    process.

    When the called routine changes the value of the parameter the original copy
    is also changed.

  \item[Pass by copy-restore (aka value-result)] \hfill \\
    The parameter value is copied to the stack of the executing process, then
    copied back when the routine terminates.

    When the called routine changes the value of the parameter the original copy
    is also changed.

\end{description}

In most situations pass by reference and pass by copy-restore yield identical
results.

In RPC pass by copy-restore is used in place of pass by reference.

\subsubsection{Packing}

Parameter marshalling and un-marshalling is converting a set of parameters to
and from a message to b sent between a client and server.

Must have a standard representation for data types, e.g. integer types, due to
different physical reorientations of different architectures, e.g. endianness.

\subsection{Binding}

\begin{description}
  \item[Static Binding] \hfill \\
    Address of server is hard coded in the client stub.
  \item[Dynamic Binding] \hfill \\
    Address of server is located at run time.
\end{description}

\subsubsection{Dynamic Binding}

\begin{itemize}
  \item The service specification is stored (registered) on a statically
        addressed binder (a.k.a. name server)
  \item The server uses the service specification to generate the server stub
  \item The developer of the client selects a suitable service specification
        which is used to compile the client stub
  \item The client stub contacts the binder at run time to obtain the address of
        the server
\end{itemize}

\subsubsection{Service Specification}

Stores following information on binder/name server:

\begin{itemize}
  \item Name of service
  \item Service routine specification
  \item Address of server running the service
\end{itemize}

Require language independent way to define service routine specification
(a.k.a function signature): Interface Definition Language (IDL).

e.g. \texttt{read(in char name[n], out char buf[n], in int length, in int pos)}

\texttt{in}, \texttt{out}, \texttt{inout} specify the direction of the
variables.

\texttt{in} and \texttt{inout} parameters are sent from the client to the
server.

\texttt{out} and \texttt{inout} parameters are sent from the server to the
client.

An \texttt{inout} parameter is the equivalent of pass by reference.

\subsection{Failure Handling}

Two possible outcomes of an RPC call:

\begin{description}
  \item[Normal Termination] \hfill \\
    Client received the reply from the server as expected.

  \item[Abnormal (Exceptional) Termination] \hfill \\
    Client does not receive a reply from the server as expected.

    This can be for a number of reasons: communication issues, node crashes,
    etc.

\end{description}

Causes of failures:

\begin{itemize}
  \item[1] Client cannot contact the server
  \item[2] Client's request message does not reach the server
  \item[3] Server's reply message does not reach the client
  \item[4] Server crashes during the call
  \item[5] Client crashes during the call
\end{itemize}

Actions in event of failures:

\begin{description}
  \item[Failure 1] \hfill \\
    Client received address unknown message from the node it tried to contact.

  \item[Failures 2, 3 \& 4] \hfill \\
    Client times out waiting for a reply from the server.

    May makes several further attempts to contact the server before an abnormal
    termination or terminate right away.

  \item[Failure 5] \hfill \\
    Orphan process is created on server.

    See section \ref{sec:rpc_orphans}.

\end{description}

Figure \ref{fig:rpc_semantics_in_errors} shows which RPC semantic is used in
each failure case.

\begin{figure}[h!]
  \centering
  \includegraphics[width=0.8\textwidth]{out/rpc_semantics_in_errors.eps}
  \caption{RPC semantics in presence of failures}
  \label{fig:rpc_semantics_in_errors}
\end{figure}

\subsubsection{Semantics}

\Para{At Least Once}

If client makes a call and it times out then it retries a finite number of
times, if it continues the fail then the client terminates abnormally.a

Server always executes the request from the client and sends the reply
regardless of previous client requests (stateless server).

\Para{At Most Once}

If a client request times out then retries are made as per At Least Once
semantics, however each request also contains a sequence number such that all
retries that are made have the same sequence number as the original request.

For each client, the server stores the results of the last executed call and the
sequence number it corresponds to (stateful server).

If a request from a client is received and its sequence number already exists in
the stored results then the stored results are returned to the client.

Need to ensure that the storage of previous results is persistent across server
reboots and crashes.

\Para{Exactly Once}

"All or nothing" behaviour requires that:

\begin{itemize}
  \item Normal termination gives exactly one execution
  \item Abnormal termination gives exactly no executions
\end{itemize}

Difficult (sometimes impossible) to implement in the presence of server crashes,
for instance if the server is in the middle of an unrecoverable operation.

Required client and server use a co-ordinated recovery strategy for handling
abnormal termination.

See atomic transactions in section \ref{sec:transactions}.

\subsubsection{Orphans}
\label{sec:rpc_orphans}

Execution started on a server when a client request is received and the client
crashes before it can receive the reply.

Orphans can consume resources and create consistency issues between orphans and
normal executions (in he case where the client retries the request).

For instance, if an RPC locks a file to be written to and the client does not
get a reply from the server they may retry by sending another RPC, which will be
waiting for the orphaned RPC. In this case neither RPC will terminate.

Treatment of orphaned execution:

\begin{itemize}
  \item When a server receives a request it could check for an identical request
        from the same client that is already being executed, if any are found
        they should be terminated before executing the current request.
  \item A server could periodically check that clients with requests executing
        on the server are alive, if the failure of a client is suspected then
        the server terminates any associated executions.
\end{itemize}

\section{Clocks and Order}
\label{sec:clocks_and_order}

TODO

\section{Transactions}
\label{sec:transactions}

For controlling operations on persistent storage resources.

"ACID" properties:

\begin{description}
  \item[Atomicity] \hfill \\
    "All or nothing". \\
    In the event of failure the state must be restored to its condition prior to
    execution.
  \item[Consistency] \hfill \\
    Ensures executions are scheduled to run such that one will not interfere
    with another.
  \item[Independence] \hfill \\
    The effects of an execution (A) must not become visible to another execution
    (B) until after A has successfully completed.
  \item[Durability] \hfill \\
    Results produced by a successful execution are not damaged by a subsequent
    failed execution.
\end{description}

TODO

\end{document}
