\documentclass[a4paper]{article}
\def\DOCTITLE{CSC8501 Advanced Programming for Games}
% Set document attributes
\title{\DOCTITLE}

\usepackage{fullpage}
\usepackage{scrextend}
\usepackage{titlesec}
\usepackage{fancyhdr}
\usepackage{amsmath}
\usepackage{amssymb}
\usepackage[section]{placeins}
\usepackage{booktabs}
\usepackage{hyperref}
\usepackage{tikz}
\usepackage{graphicx}
\usepackage{minted}
\usepackage{subcaption}

% Setup headers and footers
\pagestyle{fancy}
\lhead{}
\chead{\DOCTITLE}
\rhead{}
\rfoot{}
\cfoot{\thepage}
\lfoot{}

% New page for each section
\newcommand{\sectionbreak}{\clearpage}

% Set header and footer sizes
\renewcommand{\headrulewidth}{0.4pt}
\renewcommand{\footrulewidth}{0.4pt}
\setlength{\headheight}{15.2pt}
\setlength{\headsep}{15.2pt}

\setlength{\parskip}{5pt plus 1pt minus 1pt}
\setlength{\parindent}{0pt}

% Newline after paragraph
\newcommand{\Para}[1]{\paragraph{#1}\mbox{}}

% Stuff used in cryptography notes
\newcommand{\Forall}{\;\forall\;}
\newcommand{\Mod}{\: mod \:}

% Stuff used in distributed systems notes
\newcommand{\happenbefore}{\rightarrow}
\newcommand{\orderbefore}{\Rightarrow}
\newcommand{\clockcond}{\leadsto}
\newcommand{\RArrow}{$\rightarrow$}

\def\checkmark{\tikz\fill[scale=0.4](0,.35) -- (.25,0) -- (1,.7) -- (.25,.15) -- cycle;}


\begin{document}

\tableofcontents

\vfill
Course material:
\url{https://research.ncl.ac.uk/game/mastersdegree/programmingforgames/}

\section{Common stupidly basic things I and most other people in the class
         obviously already know but are for reasons beyond my comprehension
         asked on the exam}

\begin{itemize}
  \item
    Variables should always be initialised before being used

  \item
    Destructors in polymorphic types should be \texttt{virtual}

  \item
    Classes with pointer types need a copy constructor

\end{itemize}

\section{Polymorphism}

\begin{listing}[h!]
  \inputminted[linenos,frame=lines,firstline=5,lastline=43]{cpp}{listings/polymorphism_1.cpp}
  \caption{Polymorphism example classes}
  \label{listing:polymorphism_1}
\end{listing}
\FloatBarrier

\begin{listing}[h!]
  \inputminted[linenos,frame=lines,firstline=45]{cpp}{listings/polymorphism_1.cpp}
  \caption{Polymorphism example \texttt{main()}}
  \label{listing:polymorphism_1_main}
\end{listing}
\FloatBarrier

\begin{listing}[h!]
  \inputminted[linenos,frame=lines]{text}{out/polymorphism_1.txt}
  \caption{Polymorphism example output}
  \label{listing:polymorphism_1_out}
\end{listing}
\FloatBarrier

Notes:

\begin{itemize}
  \item
    The \texttt{virtual} keyword is required for polymorphic functions, it
    allows the vtable to be created which is used to select the implementation
    of a \texttt{virtual} function to be executed

  \item
    The closest implementation of a \texttt{virtual} function to the type of the
    instance will be called

  \item
    The implementation of a function not declared \texttt{virtual} will be that
    of the type (not the type if the instance)

  \item
    For this reason destructors should always be \texttt{virtual}

  \item
    Virtual inheritance is used to avoid the "diamond pattern" problem when a
    class inherits from multiple children of a single base class

  \item
    Without virtual inheritance this results in multiple copies of the base
    class being created

\end{itemize}

\section{Past Exam Questions}

\subsection{2013/14 Question 2}

\begin{listing}[h!]
  \inputminted[linenos,frame=lines]{cpp}{listings/csc8501_pp1314_q2.cpp}
  \caption{Sample code}
  \label{listing:csc8501_pp1314_q2}
\end{listing}
\FloatBarrier

\begin{listing}[h!]
  \inputminted[linenos,frame=lines]{text}{out/csc8501_pp1314_q2.txt}
  \caption{Output}
  \label{listing:csc8501_pp1314_q2_o}
\end{listing}
\FloatBarrier

\subsection{2013/14 Question 3}

\begin{listing}[h!]
  \inputminted[linenos,frame=lines]{cpp}{listings/csc8501_pp1314_q3.cpp}
  \caption{Sample code}
  \label{listing:csc8501_pp1314_q3}
\end{listing}
\FloatBarrier

\begin{listing}[h!]
  \inputminted[linenos,frame=lines]{text}{out/csc8501_pp1314_q3.txt}
  \caption{Output}
  \label{listing:csc8501_pp1314_q3_o}
\end{listing}
\FloatBarrier

\section{2016/17 questions}

\subsection{Question 1a}

\begin{listing}[h!]
  \inputminted[linenos,frame=lines]{cpp}{listings/csc8501_1617_q1_a.cpp}
  \caption{Sample code}
  \label{listing:csc8501_1617_q1a}
\end{listing}
\FloatBarrier

\begin{listing}[h!]
  \inputminted[linenos,frame=lines]{text}{out/csc8501_1617_q1_a.txt}
  \caption{Output}
  \label{listing:csc8501_1617_q1a_o}
\end{listing}
\FloatBarrier

\clearpage

\subsection{Question 1b}

\begin{listing}[h!]
  \inputminted[linenos,frame=lines]{cpp}{listings/csc8501_1617_q1_b.cpp}
  \caption{Sample code}
  \label{listing:csc8501_1617_q1b}
\end{listing}
\FloatBarrier

\begin{listing}[h!]
  \inputminted[linenos,frame=lines]{text}{out/csc8501_1617_q1_b.txt}
  \caption{Output}
  \label{listing:csc8501_1617_q1b_o}
\end{listing}
\FloatBarrier

\clearpage

\subsection{Question 1c}

\begin{listing}[h!]
  \inputminted[linenos,frame=lines]{cpp}{listings/csc8501_1617_q1_c.cpp}
  \caption{Sample code}
  \label{listing:csc8501_1617_q1c}
\end{listing}
\FloatBarrier

\begin{listing}[h!]
  \inputminted[linenos,frame=lines]{text}{out/csc8501_1617_q1_c.txt}
  \caption{Output}
  \label{listing:csc8501_1617_q1c_o}
\end{listing}
\FloatBarrier

\clearpage

\subsection{Question 1d}

\begin{listing}[h!]
  \inputminted[linenos,frame=lines]{cpp}{listings/csc8501_1617_q1_d.cpp}
  \caption{Sample code}
  \label{listing:csc8501_1617_q1d}
\end{listing}
\FloatBarrier

\begin{listing}[h!]
  \inputminted[linenos,frame=lines]{text}{out/csc8501_1617_q1_d.txt}
  \caption{Output}
  \label{listing:csc8501_1617_q1d_o}
\end{listing}
\FloatBarrier

\clearpage

\subsection{Question 4}

\begin{listing}[h!]
  \inputminted[linenos,frame=lines]{cpp}{listings/csc8501_1617_q4.cpp}
  \caption{Sample code}
  \label{listing:csc8501_1617_q4}
\end{listing}
\FloatBarrier

\begin{listing}[h!]
  \inputminted[linenos,frame=lines]{text}{out/csc8501_1617_q4.txt}
  \caption{Output}
  \label{listing:csc8501_1617_q4_o}
\end{listing}
\FloatBarrier

\end{document}
