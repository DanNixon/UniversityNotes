\documentclass[a4paper]{article}
\def\DOCTITLE{CSC3221 Programming For Games}
% Set document attributes
\title{\DOCTITLE}

\usepackage{fullpage}
\usepackage{scrextend}
\usepackage{titlesec}
\usepackage{fancyhdr}
\usepackage{amsmath}
\usepackage{amssymb}
\usepackage[section]{placeins}
\usepackage{booktabs}
\usepackage{hyperref}
\usepackage{tikz}
\usepackage{graphicx}
\usepackage{minted}
\usepackage{subcaption}

% Setup headers and footers
\pagestyle{fancy}
\lhead{}
\chead{\DOCTITLE}
\rhead{}
\rfoot{}
\cfoot{\thepage}
\lfoot{}

% New page for each section
\newcommand{\sectionbreak}{\clearpage}

% Set header and footer sizes
\renewcommand{\headrulewidth}{0.4pt}
\renewcommand{\footrulewidth}{0.4pt}
\setlength{\headheight}{15.2pt}
\setlength{\headsep}{15.2pt}

\setlength{\parskip}{5pt plus 1pt minus 1pt}
\setlength{\parindent}{0pt}

% Newline after paragraph
\newcommand{\Para}[1]{\paragraph{#1}\mbox{}}

% Stuff used in cryptography notes
\newcommand{\Forall}{\;\forall\;}
\newcommand{\Mod}{\: mod \:}

% Stuff used in distributed systems notes
\newcommand{\happenbefore}{\rightarrow}
\newcommand{\orderbefore}{\Rightarrow}
\newcommand{\clockcond}{\leadsto}
\newcommand{\RArrow}{$\rightarrow$}

\def\checkmark{\tikz\fill[scale=0.4](0,.35) -- (.25,0) -- (1,.7) -- (.25,.15) -- cycle;}


\begin{document}

\tableofcontents

\section{Data Types and Variables}

\subsection{Types}

A selection of data types that always seem to be on the exam:

\begin{description}
  \item[\texttt{int a}] \hfill \\
    An integer.
  \item[\texttt{int *a}] \hfill \\
    A pointer to an integer.
  \item[\texttt{char a[3]}] \hfill \\
    An array of characters of length 3.
  \item[\texttt{int *a[3]}] \hfill \\
    An array of pointers to integers of length 3.
  \item[\texttt{int (*a)[3]}] \hfill \\
    A pointer to an array of integers of length 3.
  \item[\texttt{int char *(*a)[]}] \hfill \\
    A pointer to an array of pointers to characters.
  \item[\texttt{const int * a}] \hfill \\
    A pointer to a \texttt{const} integer.
  \item[\texttt{int * const a}] \hfill \\
    A \texttt{const} pointer to an integer
\end{description}

\subsection{Declaration vs Definition}

\begin{description}
  \item[Declaration] \hfill \\
    \begin{itemize}
      \item Provides basic attributes: type and name
      \item Does not allocate storage (for variables)
      \item e.g. \texttt{extern int a;}
    \end{itemize}

  \item[Definition] \hfill \\
    \begin{itemize}
      \item Defines all attributes of a symbol
      \item e.g. \texttt{int a;}
    \end{itemize}
\end{description}

A symbol being declared but never defined results in a linker error.

\section{Polymorphism}

\begin{listing}[h!]
  \inputminted[linenos,frame=lines,firstline=5,lastline=39]{cpp}{listings/polymorphism_1.cpp}
  \caption{Polymorphism example classes}
  \label{listing:polymorphism_1}
\end{listing}
\FloatBarrier

\begin{listing}[h!]
  \inputminted[linenos,frame=lines,firstline=41]{cpp}{listings/polymorphism_1.cpp}
  \caption{Polymorphism example \texttt{main()}}
  \label{listing:polymorphism_1_main}
\end{listing}
\FloatBarrier

\begin{listing}[h!]
  \inputminted[linenos,frame=lines]{text}{out/polymorphism_1.txt}
  \caption{Polymorphism example output}
  \label{listing:polymorphism_1_out}
\end{listing}
\FloatBarrier

Notes:

\begin{itemize}
  \item The closest implementation of a \texttt{virtual} function to the type of
        the instance will be called
  \item The implementation of a function not declared \texttt{virtual} will be
        that of the type (not the type if the instance)
  \item For this reason destructors should always be \texttt{virtual}
  \item Virtual inheritance is used to avoid the "diamond pattern" problem when
        a class inherits from multiple children of a single base class
  \item Without virtual inheritance this results in multiple copies of the base
        class being created
\end{itemize}

\end{document}
